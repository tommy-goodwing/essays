

\documentclass{article}
\usepackage[utf8]{inputenc}
\usepackage{setspace}
\usepackage{ mathrsfs }
\usepackage{amssymb} %maths
\usepackage{amsmath} %maths
\usepackage[margin=0.2in]{geometry}
\usepackage{graphicx}
\usepackage{ulem}
\setlength{\parindent}{0pt}
\setlength{\parskip}{10pt}
\usepackage{hyperref}
\usepackage[autostyle]{csquotes}

\usepackage{titlesec,lipsum}



\usepackage{cancel}
\renewcommand{\i}{\textit}
\renewcommand{\b}{\textbf}
\newcommand{\q}{\enquote}
\newcommand{\p}{$\phi \ $}

%\vskip1.0in
\begin{document}
\begin{huge}
{\setstretch{0.0}{


\b{ONTOLOGICAL PERSPECTIVISM}

\section{}
\q{I am my world.} So says Wittgenstein's \q{philosophical I} in the \i{Tractatus.} His empirical ego is just a guy \i{in} the world. 

\section{}
Wittgenstein seems to be offering a new name for the old concept of \i{sakshi} or  \q{witness consciousness.}  Ontological perspectivism is essentially the claim that this \q{witness consciousness} or \q{philosophical I} is exactly the \i{being} of the world. For the ontological perspectivist, the world has no \i{other} kind of being.

\section{}
As Husserl and James both saw, what we call experience is a \q{stream.} Associated with every empirical or discursive ego is a \q{witness consciousness} which the ontological perspectivist interprets as \q{streaming world-from-perspective.} In other words, the world is a system of streamings, and we tend to understand other sentient creatures as also \q{having} the world. Somehow my wife is also a site of the world's streaming being. But she is this \q{site of being} \i{as well as} an empirical person who appears in many such worldstreamings.

\section{}
One of the obstacles to understanding ontological perspectivism is the tendency to conflate the transcendental ego and the empirical ego. Note that \q{transcendental ego} is yet another name for the concept behind both the \q{philosophical I} and \q{witness consciousness.} In all cases the \q{emptied subject} shrinks to a point, in fact vanishes. The confusion of the two egos is natural enough, especially if one understands belief as a snapshot of the conceptual structure of a worldstreaming (of a \q{desubjectivized} philosophical I). The empirical ego is also and even especially in this context a linguistic ego who makes claims in a normative space. These claims, when sincere, are articulations of belief, which is to say of the current intelligible structure of the worldstreaming associated with that linguistic ego.  

\section{}
Let me summarize with a statement that may sound absurd: The world is a system of streams, in which \i{consciousness does not exist.} More exactly, the consciousness of the hard problem does not exist. The practical concept is still useful. But there is no radical gap between the purely mental and purely physical. There is no \q{thing in itself} or \q{aperspectival residue.} 

\section{}
How does the ontological perspectivist, while denying dualism, deal with daydreams, for instance ? Inferentialism comes to the rescue here. I can use a \q{private} daydream as an example in a philosophical argument (to justify a claim.) I can blame this daydream on a molecule, by saying that it was a side-effect of a pain-killer. I can then blame a minor car accident on this vivid daydream. The point is that we \i{already} move comfortably between the realm of the \q{private} mental and the public physical. 

\section{}
We might therefore choose to talk in terms of differential access. It is \q{your} toothache, so you have a different \q{first-person} or \q{direct} access to it. But I, as another member in the forum or public space of reasons, still have indirect access to it. Toothache is a pubic concept, governed by inferential norms. 

\section{}
If this differential access seems like a trick, let us consider the access of a completely blind person to a painting. They can touch it. They can talk about it. Their access is different, but the painting exists in the public space of reasons. It has, most crucially, a \q{logical} being, which is to say a place in the \q{game} of giving and asking for reasons. 


\section{}
Since hallucinations are also clearly in this space of reasons (intelligible in rational discourse), they are real in some more inclusive sense of the word. ( To be sure, practical conversation will exclude things like daydreams and hallucinations from the label.)  

\section{}
If we follow this insight, we can understand the appearance/reality distinction as merely practical and not ontologically necessary. In fact, this appearance-reality distinction is exactly the dualism ontological perspectivism transcends (not practically, of course, since we don't forget the usual way of talking, but theoretically.)

\section{}
If we consider worldstreamings again, we understand them to contain both daydreams and dandruff (and primes and pumpkins.) Note that the worldstream is phenomenology's lifeworld. It is always already significant. It includes the equipmental nexus discussed by Heidegger. It is \q{like} (just like) the \q{total experience} of the associated empirical ego or person (or puppy). But we eventually drop the word \q{experience} as too subjective for what is in fact just a piece or a side of the flowing world --- a single world which is streamed  perspectively (relative the bodies of the associated sentient creatures.) 


 }}
\end{huge}
\end{document}
